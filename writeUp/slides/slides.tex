\documentclass[11pt]{beamer}
\usetheme{Warsaw}
\usepackage[utf8]{inputenc}
\usepackage{amsmath}
\usepackage{amsfonts}
\usepackage{amssymb}
%\author{}
%\title{}
%\setbeamercovered{transparent} 
%\setbeamertemplate{navigation symbols}{} 
%\logo{} 
%\institute{} 
%\date{} 
%\subject{} 
\begin{document}

%\begin{frame}
%\titlepage
%\end{frame}

%\begin{frame}
%\tableofcontents
%\end{frame}

\title{Lung nodule analysis using deep learning}
\author{Matt Smith}
\institute{University of Warwick}
\date{\today}


\begin{frame}
\frametitle{Outline}
\tableofcontents
\end{frame}

\section{Section 1}
\subsection{Background}
\begin{frame}
\frametitle{Background}
\begin{itemize}
\item Lung cancer is the second-most common type of cancer and the leading cause of cancer death in the United States \cite{acs}.
\item  Computed tomography images (CT) are used in radiotherapy which contrast soft and hard tissues to allow precise target delineation of candidate nodules.
\item  Computer-Aided Detection (CAD) systems aid radiologists by performing a nodule detection pass.
\begin{itemize}
\item Large number of false positives.
\item Not 100\% sensitive.
\end{itemize}
\end{itemize}
\end{frame}

\subsection{Deep learning}
\begin{frame}
\frametitle{Deep learning}
\begin{itemize}

\item Deep learning (Artificial Neural Networks) are function approximators.
\item Model highly complex relationships between inputs and outputs with great results.
\begin{itemize}
\item Requires lots of data.
\item Hard to train/generalize.
\end{itemize}

\end{itemize}
\begin{itemize}
\item Large number of applications in multiple industries.
\end{itemize}
\end{frame}

\subsection{Convolutional Neural Networks}
\begin{frame}
\frametitle{Convolutional Neural Networks (CNNs)}
\begin{itemize}
\item Deep learning algorithms have two main subsets; 
\begin{itemize}
\item Recurrent
\begin{itemize}
\item Have internal memory and modelling sequences.
\end{itemize}
\item Convolutional
\begin{itemize}
\item Highly effective with temporal, spatial or volumetric patterns.
\end{itemize}
\end{itemize}

\item Motivation for lung nodule analysis:
\begin{itemize}
\item 2D CNNs are state of the art in computer vision problems.
\item 3D CNN infrastructure is being laid down in machine learning libraries e.g. Torch, Tensorflow, Theano.
\item Lots of 3D CT scan data available.
\item Computational power improvements.
\end{itemize}
\end{itemize}
\end{frame}


\section{Section 2}
\subsection{Data}
\begin{frame}
\frametitle{LUNA16 (Lung nodule analysis 2016)}
\begin{itemize}
\item 888 CT patient scans.

\begin{itemize}
\item Computer aided diagnosis algorithms generate candidates with $x,y,z$ coordinates.
\begin{itemize}
\item 754,975 candidates across scans.
\item 1,557 true positive nodules.
\item 753,418 false positive nodules.
\item 500:1 FP:TP ratio.
\item 1,186 true positives have a diameter included.
\end{itemize}
\end{itemize}
\end{itemize}
\begin{center}
\includegraphics[scale=0.25]{../noduleDiameter.jpg}
\end{center}

\end{frame}
\begin{frame}
\begin{center}
\includegraphics[scale=0.5]{../lung.png}
\end{center}


\end{frame}



\end{document}
