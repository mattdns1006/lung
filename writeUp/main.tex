\documentclass{article}%%%%where IET is the template name

%The authors can define any packages after the \documentclass{IET} command.

\newtheorem{condition}{Condition}

\newcommand{\fourPic}{0.225}

\begin{document}

\title{Lung Nodule Analysis}
\maketitles

\subsection{Introduction}

Lung cancer is the second-most common type of cancer and the leading cause of cancer death in the United States \cite{acs}. Computed tomography images (CT) are widely used in radiotherapy as they output good resolution 3-D images which contrast soft and hard tissues to allow precise target delineation of candidate nodules. Survival after a late diagnosis is uncommon and is directly proportional with its growth after detection time \cite{dimililer2017tumor}. Computer-Aided Detection (CAD) systems are designed to aid radiologists by  performing a nodule detection pass. Large numbers of false positives are generated by current lung nodule segmentation systems, usually such that the ratio of false positives to actual positives can be in the hundreds. Moreover CAD systems are not 100\% sensitive and can generate some false negatives requiring a full check by the radiologist anyway. Not only is this especially time consuming for radiologists but it can lead to errors in diagnosis. 

\subsection{Related work}


Conventional automated medical imaging diagnosis or  has had a recent shift towards the use of Convolutional Neural Networks (CNNs) \cite{bush2016lung, ramaswamypulmonary, shin2016deep} due to availability of vast amounts of data, improvements in computational power and optimization and their superior performance in many image recognition challenges \cite{2015imagenet,simonyan2014very,taigman2014deepface}.

Most applications of CNNs to CT scan data use a 2-D structure, i.e. exploiting spatial information in each of the triplanar views seperately (sagittal, coronal and axial planes). \cite{setio2015automatic} seperate the region of interest into the 9 symmetrical planes of a cue before being fed into a 2-D network. 

Furthermore these 2-D networks are normally pretrained on non medical image tasks in order to extract features for classification (transfer learning) \cite{ramaswamypulmonary}.

3-D CNNs can be used to capitalize on volumetric features of nodules. Air tracts for exampe, are hard to differentiate with nodules in 2-D yet easier in 3-D and hence are currently a primary cause for false positives \cite{anirudh2016lung}. To the best of our knowledge they have only recently been shown to work well in false positive reduction by \cite{dou2016multi}.

Whilst it is useful to classify nodules it is also of interest for medical practitioners to localize nodules, for example creating a bounding cube around the localized region.

\section{Classification}

\subsubsection{Results}

\section{Conclusion}

\bibliography{mybib}
\bibliographystyle{abbrv}

\end{document}
